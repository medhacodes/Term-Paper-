\documentclass[12pt]{article}

\usepackage{graphicx}
\graphicspath{{images}}
\usepackage{adjustbox}
\begin{document}

\tableofcontents
\clearpage

\section*{ABSTRACT}
Neuralink is a device that will be surgically inserted into the brain using robotics by neurosurgeons. This device can then be used to operate smartphones and computers without having to touch it. In this procedure, a chip-set called the link is implanted in the skull. Neuralink is an implant that reads electrical signals produced by neurons in the brain. The device is implanted directly into the brain because placing it outside the head will not detect the signals accurately. Neurons are made up of three parts called the dendrite, soma and axon. Neuralink is a connection between the human brain and the internet. This means that people with paralysis can easily operate their phones and computer directly with their brain. Its main purpose is to help people to communicate through text or voice messages. It can also be used to draw pictures, take photographs and do other activities.
By 2025, technology like neural lace will be widely available to everyone. You'll be able to seamlessly communicate with other people without them ever knowing you're using this technology.
In this term paper let us discuss more about neuralink and its future aspects.
\clearpage

\section{INTRODUCTION}

Neuralink is a device that will be surgically implanted in your brain, allowing you to connect with and even control machines. It will also aid in the study of brain electrical signals and the development of solutions to various medical conditions. Since its founding in 2016, the company has been focused on the development of this technology. 
With Neuralink, an 8mm-diameter chipset called the N1 chipset will be implanted in our skull, with several wires containing electrodes and insulation for the wires. A robot will surgically implant these wires into our brain. According to the company, the wire is as thick as our brain's neurons and as thin as a strand of hair at 100 micrometres. To make a comparison, picture the diameter of our hair and multiply it by ten.
Max Hodak, president of Neuralink, says that we can place more than one device to target different sections of our brain.

\begin{figure}
\centering
\includegraphics[scale=0.4]{Image1}
\caption{}
\end{figure}
\clearpage

\section{WORKING OF NEURALINK}

It's best to grasp the science behind the human brain before learning how Neuralink works. Neurons in the brain provide information to cells throughout the body, including muscle, nerve, gland, and other neuron cells. The dendrite, soma (cell body), and axon are the three elements that make up a neuron. Each component has a distinct purpose. The impulses are received by the dendrite. These impulses are processed by the soma. The signals are subsequently transmitted to the other cells through the axon. Synapses, which produce neurotransmitters, connect the neurons to one another. These chemical compounds are then sent to the dendrite of another neuron cell, causing current to flow across the neurons. The electrodes that make up the Neuralink will read electrical impulses produced by a number of neurons in the brain. After then, the signals are output in the form of an action or movement. According to the company's website, the gadget is implanted directly in the brain since it cannot correctly detect the signals produced by the brain if it is placed outside the head. Now that you understand what Neuralink is and how it works, learn what Neuralink performs.

\begin{figure}
\centering
\includegraphics[scale=0.4]{Image2}
\caption{Figure}
\end{figure}


\clearpage

\section{BENEFITS OF NEURALINK}
\begin{itemize}

\item Neuralink's advantages include the ability to treat encephalopathy.
\item It could potentially be employed as a link between humans and technology.
\item	People who are paralysed may benefit from being able to control their phones directly with their minds.
\item	It can also be used to draw, take photos, and so on.
\item	This technique has the potential to be a game-changer for patients with numerous neuro diseases and brain illnesses.
\item	If Elon Musk succeeds in this endeavour, the world will be a much better place. Many people who have been paralysed have contacted Musk, requesting that he utilise them as subjects in his research, but Musk insists on first testing it on monkeys.
\end{itemize}


\begin{figure}
\centering
\includegraphics[scale=0.4]{Image3}
\caption{}
\end{figure}

\clearpage

\section{DISADVANTAGES OF NEURALINK}
\begin{itemize}

\item Brain surgery is less dangerous than we might anticipate if done carefully and under sterile settings.
\item Decomposition of electrode materials by the body over time may have some unknown long-term effects on the brain.
\item Bluetooth device that can be controlled by a computer that is directly connected to your brain. It's critical that the security is top-notch.
\end{itemize}
\begin{figure}
\centering
\includegraphics[scale=0.4]{Image4}
\caption{Figure}
\end{figure}

\section{CONCLUSION}

After the utah array neuralink is another and upgraded breakthrough in neural engineering. Neuralink may be the bridge that connects human beings to the next level of artificial intelligence but many people might have reservations on a computer chipset inside their brains. The neuralink is successfully implanted in pig but as of now, the technology has not been tested on humans. The vision of this technology can be fulfilled when neuralink is successfully implanted in humans. Neuralink can be the biggest invention/researches of the century if it is successfully implanted in humans. Neuralink chip also study the electrical signals in the brains and arrive at solutions that can help to cure so many lives suffering from neural disabilities or various medical problems.
\clearpage

\section{REFERENCES}
\begin{itemize}
\item neuralink.com
\item wikipedia.org/wiki/Neuralink
\item youtube.com

\end{itemize}

\end{document}